\documentclass[12pt]{article}
\usepackage[a4paper]{geometry}
\usepackage[utf8]{inputenc}
\geometry{tmargin=2cm,bmargin=2cm,lmargin=2cm,rmargin=2cm}
\usepackage[main=russian, english]{babel}
\usepackage{graphicx}
\usepackage{amsmath}
\pagenumbering{gobble}
Определения:

\begin{itemize}
\item Нота абстрактная --- (октава, ступень, альтерация).
\item Аккорд --- массив из 4-х нот, заданных по определённым правилам.
\end{itemize}

\begin{document}
\begin{enumerate}
\item Ноты должны принадлежать одной из 3-х функций: T, D, S.
\item Аккорды могут быть заданы в двух расположениях: тесном и широком.
\item После аккорда D не может быть аккорда S.
\item Интервал между басом и тенором может быть не более 2-х октав.
\item Интервал межды остальными соседними голосами может быть не более октавы.
\item В соседних аккордах все голоса не могут двигаться в одну сторону или оставаться на месте.
\item На первой доле нового такта не может повторяться функция, которая была на последней доле предыдущего такта.
\item В случае, если в соседних аккордах одинаковые функции, и известная нота прнадлежит другому расположению аккорда, разрешена смена расположения: с широкого на тесное, и наоборот.
\item Запрещены параллельные квинты, и параллельные октавы. В соседних аккордах если интервал между какой-либо парой нот составляет октаву либо квинту, во втором аккорде интервал не может составлять октаву либо квинту соответственно.
  Параллельные октавы мы можем вычислить либо по формуле:
  \begin{equation}
    \begin{aligned}
      (N_i[0] - N_j[0]) \mod 7 = 0\\
      (N_i[1] - N_j[1]) \mod 7 = 0
    \end{aligned}
  \end{equation}
  Либо по совпадению ступеней: {\tt note(\_, step1), note(\_, step1)} и в последующем такте: {\tt note(\_, step2), note(\_, step2)}.

  Параллельные квинты:
\end{enumerate}
\end{document}
