\documentclass[12pt]{article}
\usepackage[a4paper]{geometry}
\usepackage[utf8]{inputenc}
\geometry{tmargin=2cm,bmargin=2cm,lmargin=2cm,rmargin=2cm}
\usepackage[main=russian, english]{babel}
\usepackage{graphicx}
\usepackage{amsmath}
\pagenumbering{gobble}
Определения:

\begin{itemize}
\item Нота абстрактная --- (октава, ступень, альтерация).
\item Аккорд --- массив из 4-х нот, заданных по определённым правилам.
\end{itemize}

\begin{document}
\begin{enumerate}
\item Ноты должны принадлежать одной из 3-х функций: T, D, S.
\item Аккорды могут быть заданы в двух расположениях: тесном и широком.
\item После аккорда D не может быть аккорда S.
\item Интервал между басом и тенором может быть не более 2-х октав.
\item Интервал межды остальными соседними голосами может быть не более октавы.
\item В соседних аккордах все голоса не могут двигаться в одну сторону или оставаться на месте.
\item На первой доле нового такта не может повторяться функция, которая была на последней доле предыдущего такта.
\item В случае, если в соседних аккордах одинаковые функции, и известная нота прнадлежит другому расположению аккорда, разрешена смена расположения: с широкого на тесное, и наоборот.
\item Запрещены параллельные квинты, и параллельные октавы. В соседних аккордах если интервал между какой-либо парой нот составляет октаву либо квинту, во втором аккорде интервал не может составлять октаву либо квинту соответственно.
  Параллельные октавы мы можем вычислить либо по формуле:
  \begin{equation}
    \begin{aligned}
      (N_i[0] - N_j[0]) \mod 7 = 0\\
      (N_i[1] - N_j[1]) \mod 7 = 0
    \end{aligned}
  \end{equation}
  Либо по совпадению ступеней: {\tt note(\_, step1), note(\_, step1)} и в последующем такте: {\tt note(\_, step2), note(\_, step2)}.

  Параллельные квинты:
\end{enumerate}
\begin{minted}[breaklines, breakanywhere, tabsize=2]{prolog}
?- getMusicFromXML('hello3.xml', maj, Notes, music_attrs(Key, Beats)), getChords(Key, Notes, Chords).

Notes = [note(ts(0, 0), voice(1), duration(240), pitch(4, 'C', 0)), note(ts(0, 240), voice(1), duration(240), pitch(4, 'E', 0)), note(ts(0, 480), voice(1), duration(240), pitch(4, 'G', 0)), note(ts(1, 0), voice(1), duration(240), pitch(4, 'D', 0)), note(ts(1, 240), voice(1), duration(240), pitch(4, 'F', 1)), note(ts(1, 480), voice(1), duration(240), pitch(4, 'B', 0))],
Key = key(4, 0, maj),
Beats = beats(3, 720),

Chords = [
  chord([stage(3, 3, 0), _1008, _1014, _1020], ts(0, 0), duration(240), type(_998)),
  chord([stage(3, 5, 0), _1428, _1434, _1440], ts(0, 240), duration(240), type(_1418)),
  chord([stage(4, 0, 0), _3246, _3252, _3258], ts(0, 480), duration(240), type(_3236)),
  chord([stage(3, 4, 0), _5288, _5294, _5300], ts(1, 0), duration(240), type(_5278)),
  chord([stage(3, 6, 0), _7348, _7354|...], ts(1, 240), duration(240), type(_7338)),
  chord([stage(4, 2, 0), _9408|...], ts(1, 480), duration(240), type(_9398))].
\end{minted}
\end{document}
